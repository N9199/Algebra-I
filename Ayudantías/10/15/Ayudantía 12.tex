\documentclass{ayudantia}

\usepackage{multicol}

\title{Ayudantía 12}
\date{2019/10/15}
\author{Nicholas Mc-Donnell}
\course{Álgebra I - MAT2227}

\begin{document}
\maketitle
\begin{itemize}
    \item[0)] Escriba la definición de Ideal, Ideal principal, Ideal Maximal, etc.
\end{itemize}
\begin{enumerate}
    \item Sea \(R\) un anillo unitario e \(I\) ideal principal generado por \(a\), demuestre que todo elemento en \(I\) es de la forma \(ra\) con \(r\in R\).
    \item Sea \(R\) un anillo unitario e \(I\) un ideal generado por dos elementos \(a,b\), demuestre que todo elemento en \(I\) es de la forma \(ra+sb\) con \(r,s\in R\). Más generalmente, sea \(I\) un ideal generado por finitos elementos \(a_1,\dots,a_n\), demuestre que todo elemento en \(I\) es de la forma \(r_1a_1+\dots+r_na_n\) con \(r_i\in R\) \(\forall 1\leq i\leq n\)
    \item Sea \(R\) un anillo unitario tal que todo elemento salvo el \(0\) es unidad\footnote{i.e. tiene inverso}\footnote{\(R\) es un cuerpo}, demuestre que solo existen los ideales triviales (el generado por \(0\) y todo el anillo)
    \item Demuestre que todo ideal generado por algún elemento irreducible es maximal.
    \item Demuestre que todo ideal en \(\set{Z}\) es principal.
\end{enumerate}
\end{document}