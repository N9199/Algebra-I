\documentclass{ayudantia}

\usepackage{multicol}

\title{Ayudantía 11}
\date{2019/10/10}
\author{Nicholas Mc-Donnell}
\course{Álgebra I - MAT2227}

\begin{document}
\maketitle
\begin{itemize}
    \item[0)] Escriba la definición de Anillo
\end{itemize}
\begin{enumerate}
    \item Demuestre que los siguientes conjuntos son anillos con las operaciones usuales:
    \begin{multicols}{4}
        \begin{enumerate}[label=(\alph*)]
            \item \(\set{Z}\)
            \item \(\set{Q}\)
            \item \(\set{R}\)
            \item \(\set{C}\)
            \item \(\set{Z}_n\)
            \item \(\set{Z}[x]\)
            \item \(\set{Q}[x]\)
            \item \(\set{R}[x]\)
            \item \(\set{C}[x]\)
            \item \(\set{Z}_n[x]\)
            \item Los pares
            \item \(\{0\}\)
        \end{enumerate}
    \end{multicols}
    \item Demuestre que dado un anillo \(R\), \(R[x]\) los polinomios con coeficientes en \(R\) es un anillo.
    \item Dado dos anillos \(R\) y \(S\), se definen las siguientes operaciones:
    \begin{align*}
        \cdot:&(R\times S)\times(R\times S)\rightarrow R\times S\\
        &(r_1,s_1)\cdot(r_2,s_2)\mapsto(r_1\cdot r_2,s_1\cdot s_2)\\
        +:&(R\times S)\times(R\times S)\rightarrow R\times S\\
        &(r_1,s_1)+(r_2,s_2)\mapsto(r_1+r_2,s_1+s_2)
    \end{align*}
    Demuestre que \(R\times S\) es un anillo con esas operaciones.
    \item Sea \(S\) un conjunto, demuestre que \((P(S),\Delta,\cap,\emptyset)\)\footnote{\(\Delta\) es la diferencia simetrica y se toma como la suma, \(\cap\) es la intersección y se toma como la multiplicación} es un anillo.
    \item Sea \(S\) un conjunto y \(R\) un anillo, demuestre que el conjunto de funciones de \(S\) a \(R\) es un anillo con la suma punto a punto y la multiplicación punto a punto.
\end{enumerate}
\end{document}