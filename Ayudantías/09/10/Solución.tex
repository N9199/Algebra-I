\documentclass{ayudantia}

\usepackage{multicol}

\title{Solución Ayudantía 6}
\date{2019/09/10}
\course{Álgebra I - MAT2227}

\begin{document}
\maketitle
\begin{enumerate}
    \item \begin{enumerate}
        \item Se nota que \((3,11)=1\) y que \(\varphi(11)=10\) por lo que \(3^{200}\equiv(3^{10})^20\equiv 1\mod 11\).
        \item Se nota que \((7,12)=1\) y que \(\varphi(12)=4\), por lo que \(7^{256}\equiv (7^4)^{64}\equiv 1\mod 12\).
        \item Se nota que \((4,9)=1\) y \(\varphi(9)=6\), como \(6\mid 9072\), se tiene que \(4^{9072}\equiv 1\mod 9\).
    \end{enumerate}
    \item Se nota que ver el dígito de la unidad de un número es equivalente a ver el número modulo 10.
    \begin{enumerate}
        \item Como \((3,10)=1\) y \(\varphi(10)=4\), entonces \(3^{90}\equiv3^{88}\cdot 3^2\equiv 9 mod 10\), por lo que el dígito de la unidad es \(9\).
        \item Como \((17,10)=1\) y \(4\mid 212\), entonces \(17^{212}\equiv 1\mod 10\).
        \item Se escribe \(9!=9\cdot 8\cdot 7\cdot 6\cdot 5\cdot 4\cdot 3\cdot 2\cdot 1\), y se nota que \(10\mid 9!\), por lo que su dígito de la unidad es \(0\).
        \item Se escribe el producto modulo 10: \(2\cdot 3\cdot 7\cdot 11\cdot 13\cdot 17\cdot 19\cdot 23\equiv 2\cdot 3^3\cdot 7^2\cdot 9\equiv 3^5\cdot (-3)^2\equiv 3^7\equiv 3^3\equiv 27\equiv 7\mod 10\).
    \end{enumerate}
    \item Sea \(n\) minimal tal que \(a^n\equiv 1\mod 36\), con \(a\) fijo y coprimo con \(36\). Luego se nota que \(\varphi(36)=12\), por lo que \(n\leq 12\). Ahora se asume que \(n\nmid 12\), por lo \(12=n\cdot q+r\), donde \(0<r<n\), luego \(a^{12}\equiv a^{n\cdot q+r}\equiv (a^{n})^q\cdot a^r\equiv a^r\equiv 1\mod 36\), por lo que \(n\) no es minimal. Se nota que esta demostración implica que los \(n\) minimales tal que \(a^n\equiv 1\mod m\) para algún \(a\) fijo coprimo con \(m\), cumplen que \(n\mid \varphi(m)\).
    \item Como \(p\mid m\), se tiene que \(a^{m-1}\equiv 1\mod p\), como \(p\) primo, se tiene que el mínimo \(n\) tal que \(a^n\equiv 1\mod p\) es \(p-1\), por lo que \(p-1\mid m-1\). Similarmente se tiene que \(\varphi(p^r)\mid m-1\), pero se nota que \((m-1,m)=1\), y como \(p\mid m\), se tiene que \(r=1\)\footnote{\(\varphi(p^r)=p^{r-1}(p-1)\)} y más aún se tiene que \(p\neq 2\), por lo mismo.
\end{enumerate}

\end{document}