\documentclass{ayudantia}

\title{Ayudantía 7}
\date{2019/09/12}
\author{Nicholas Mc-Donnell}
\course{Álgebra I - MAT2227}

\begin{document}
\maketitle
\begin{enumerate}
    \item Calcule los siguientes valores:
    \begin{enumerate}[label=(\alph*)]
        \item \(3^{513}\mod 16\)
        \item \(2^{8765}\mod 12\)
        \item \(k^{301}\mod 9\) con \((3,k)=1\)
        \item \(5^{n^2}\mod 8\) con \((2,n)=1\)
    \end{enumerate}
    \item Dado \(p>1\) entero, demuestre que las siguientes condiciones son equivalentes:
    \begin{enumerate}[label=(\alph*)]
        \item \(p\) es primo.
        \item Para cualquier \(a\in\set{Z}_p\setminus\{0\}\), la congruencia \(ax\equiv 1\mod p\) tiene una solución.
        \item Si \(ab\equiv 0\mod p\), entonces \(a=0\) o \(b=0\).
    \end{enumerate}
\end{enumerate}
\end{document}