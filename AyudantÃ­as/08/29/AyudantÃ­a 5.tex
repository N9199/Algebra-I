\documentclass{ayudantia}

\title{Ayudantía 5}
\date{2019/08/29}
\author{Nicholas Mc-Donnell}
\course{Álgebra I - MAT2227}

\begin{document}
\maketitle
\begin{enumerate}
    \item Encuentre las soluciones de las siguientes sistemas de congruencias:
    \begin{enumerate}
        \item \(x\equiv 5y\mod 6\), \(3x\equiv 7\mod 10\)
        \item \(x\equiv 2\mod 3\), \(x\equiv 5\mod 7\), \(x\equiv 6\mod 8\)
        \item \(3(x-2y)\equiv 2(y-3)\mod 25\), \(3(x-y)\equiv 4(5y-2)\mod 49\)
    \end{enumerate}
    \item Demuestre que dados \(a,b,x\) enteros tales que \(x^a\equiv 1\mod m\) y \(x^b\equiv 1\mod m\), entonces se tiene  que \(x^{(a,b)}\equiv 1\mod m\)
    \item Demuestre que dados \(x,y\) tales que \(x\equiv y\mod a\) y \(x\equiv y\mod b\), entonces \(x\equiv y\mod\mcm(a,b)\)
    \item Muestre que dado \(x,y\) con las mismas condiciones anteriores, no necesariamente \(x\equiv y\mod ab\)
    \item Encuentre todos los números naturales \(n,k\) tales que \(\sum_{i=1}^ni=10^k\)
\end{enumerate}
\end{document}