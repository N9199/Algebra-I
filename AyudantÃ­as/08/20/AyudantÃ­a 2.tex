\documentclass{ayudantia}

\title{Ayudantía 2}
\date{2019/08/20}
\course{Álgebra I - MAT2227}

\begin{document}
\maketitle
\begin{enumerate}
    \item \(a^2\mid b^2\) ssi \(a\mid b\)
    \item \((a,b)=1\) entonces \((a+b,ab)=1\)
    \item Recordando que un entero se escribe en notación decimal como \(a_na_{n-1}\dots a_0\) donde \(a_0\)
    son las unidades, \(a_1\) las decenas, etc. Demuestre los siguientes criterios de divisibilidad:
    \begin{enumerate}
        \item Un número es divisible por 2 si \(2\mid a_0\)
        \item Un número es divisible por 3 si la suma de sus dígitos es divisible por 3
        \item Un número es divisible por 4 si \(4\mid a_1a_0\). El número también divisible por 4 si \(4\mid 2a_1+a_0\)
        \item Un número es divisible por 5 si \(5\mid a_0\)
    \end{enumerate}
    \item Escriba un algoritmo que enumere todos los primos (Demuestre que funciona)
\end{enumerate}

\end{document}