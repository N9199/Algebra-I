\documentclass{ayudantia}

\title{Ayudantía 6}
\date{2019/09/10}
\author{Nicholas Mc-Donnell}
\course{Álgebra I - MAT2227}

\begin{document}
\maketitle
\begin{enumerate}
    \item Calcule los siguientes valores:\\
          Ejemplo: \(2^4\mod 5\), se sabe que \(2^4=16\), por lo que \(2^4\equiv 1\mod 5\).
          \begin{enumerate}
              \item \(3^{200}\mod 11\)
              \item \(7^{256}\mod 12\)
              \item \(4^{9072}\mod 9\)
          \end{enumerate}
    \item Encuentre el dígito de la unidad de los siguientes números:
    \begin{enumerate}
        \item \(3^{90}\)
        \item \(17^{212}\)
        \item \(9!\)
        \item \(2\cdot 3\cdot 7\cdot 11\cdot 13\cdot 17\cdot 19\cdot 23\)
    \end{enumerate}
    \item Demuestre que el menor número natural \(n\) tal que \(a^n\equiv  1\mod 36\) con \(a\) fijo y coprimo con 36, es \(1,2,3,4,6\) o \(12\).
    \item Sea \(m\geq3\) un entero tal que para todo \(a\in\set{Z}\) con \((a,m)=1\) se cumple
          \[a^{m-1}\equiv 1\mod m\]
          Sean \(p\) un número primo y \(r\geq 1\) tal que \(p^r\mid m\). Demuestre que \(r=1\), \(p\geq3\) y que \(p-1\mid m-1\)
\end{enumerate}
\end{document}